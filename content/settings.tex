%%%%%%%%%%%%%%%%%%%%%%%%%%%%%%%%%%%%%%%%%%%%%%%%%%%%%%%%%%%%%%%%%%%%%%%%%%%%%%%
%                                   Einstellungen
%
% Hier können alle relevanten Einstellungen für diese Arbeit gesetzt werden.
% Dazu gehören Angaben u.a. über den Autor sowie Formatierungen.
%
%
%%%%%%%%%%%%%%%%%%%%%%%%%%%%%%%%%%%%%%%%%%%%%%%%%%%%%%%%%%%%%%%%%%%%%%%%%%%%%%%


%%%%%%%%%%%%%%%%%%%%%%%%%%%%%%%%%%%% Sprache %%%%%%%%%%%%%%%%%%%%%%%%%%%%%%%%%%
%% Aktuell sind Deutsch und Englisch unterstützt.
%% Es werden nicht nur alle vom Dokument erzeugten Texte in
%% der entsprechenden Sprache angezeigt, sondern auch weitere
%% Aspekte angepasst, wie z.B. die Anführungszeichen und
%% Datumsformate.
\setzesprache{de} % de oder en
%%%%%%%%%%%%%%%%%%%%%%%%%%%%%%%%%%%%%%%%%%%%%%%%%%%%%%%%%%%%%%%%%%%%%%%%%%%%%%%


%%%%%%%%%%%%%%%%%%%%%%%%%%%%%%%%%%% Angaben  %%%%%%%%%%%%%%%%%%%%%%%%%%%%%%%%%%
%% Die meisten der folgenden Daten werden auf dem
%% Deckblatt angezeigt, einige auch im weiteren Verlauf
%% des Dokuments.
\setzemartrikelnr{1234567}
\setzekurs{TINF12ITA}
\setzetitel{Thema der Arbeit}
\setzedatumAnfang{01.01.2023}
\setzedatumAbgabe{31.12.2023}
\setzefirma{Firma GmbH}
\setzefirmenort{Stuttgart}
\setzeabgabeort{Stuttgart}
\setzeabschluss{Bachelor of Science}
\setzestudiengang{IT-Automotive}
\setzedhbw{Stuttgart}
\setzehochschule{Duale Hochschule Baden-Württemberg}
\setzebetreuer{B. Sc. Max Mustermann}
\setzegutachter{Prof. Dr. rer. nat. Gustaf Gutachter}
\setzezeitraum{\datumAnfang\ - 14.08.2023}
\setzearbeit{T3\_1000}
\setzeautor{John Doe}
\setzesemester{1.}
\setzestudienrichtung{Informatik}
\setzejahrgang{2022}
\setzeabteilung{Abteilung}
\setzestandort{Stuttgart}
\setzefusszmitte{\arbeit}
\setzekopfz{Firma GmbH\\IT-Automotive} % wird angezeigt, falls eins oder beide Logos nicht gesetzt sind
\setzesingingdate{\today} % adjust the date that appears on the blocking notice and the declaration -> \today for date of compilation


\sperrvermerktrue           % auskommentieren oder ändern zu \sperrvermerkfalse, wenn kein Sperrvermerk benötigt wird
\selbsterkltrue             % auskommentieren oder ändern zu \selbsterklfalse, wenn keine Selbstständigkeitserklärung benötigt wird
\abstractdetrue             % auskommentieren oder ändern zu \abstractdefalse, wenn kein deutscher Abstract gewünscht ist
\abstractentrue             % auskommentieren oder ändern zu \abstractenfalse, wenn kein englischer Abstract gewünscht ist
\inhalttrue                 % auskommentieren oder ändern zu \inhaltfalse, falls kein Inhaltsverzeichnis eingefügt werden soll
\abkverztrue                % auskommentieren oder ändern zu \abkverzfalse, wenn kein Abkürzungsverzeichnis benötigt wird
\abbverztrue                % auskommentieren oder ändern zu \abbverzfalse, wenn kein Abbildungsverzeichnis benötigt wird
\tableverztrue              % auskommentieren oder ändern zu \tableverzfalse, wenn kein Tabellenverzeichnis benötigt wird
\formelgroeverztrue			% auskommentieren oder ändern zu \formelgroeverzfalse, wenn kein Formelgrößenverzeichnis benötigt wird
\formelverztrue             % auskommentieren oder ändern zu \formelverzfalse, wenn kein Formelverzeichnis benötigt wird
\codeverztrue               % auskommentieren oder ändern zu \codeverzfalse, wenn kein Listingsverzeichnis benötigt wird
\appendixtrue               % auskommentieren oder ändern zu \appendixfalse, wenn die folgenden drei Einstellungen nicht gewünscht sind
\literaturtrue              % auskommentieren oder ändern zu \literaturfalse, wenn kein Literaturverzeichnis gewünscht ist (\appendixtrue muss gesetzt sein!)
\glossartrue                % auskommentieren oder ändern zu \glossarfalse, wenn kein Glossar gewünscht ist (\appendixtrue muss gesetzt sein!)
\anhangtrue                 % auskommentieren oder ändern zu \anhangfalse, wenn kein Anhang gewünscht ist

%\refWithPagestrue           % ändern zu \refWithPagestrue, wenn die Seitenzahl bei Verweisen auf Kapitel engefügt werden sollen

\reviewertrue	            % auskommentieren oder ändern zu \reviewerfalse wenn kein Gutachter gesetzt werden muss

% Angabe des roten/gelben/grünen Punktes auf dem Titelblatt zur Kennzeichnung der Vertraulichkeitsstufe.
% Mögliche Angaben sind \greendottrue, \yellowdottrue und \reddottrue. Werden mehrere angegeben, wird der Punkt mit der höheren
% Vertraulichkeitsstufe angezeigt.
% Wird keines der Kommandos angegeben, wird kein Punkt gezeichnet.
%\greendottrue
%\yellowdottrue
%\reddottrue

%%%%%%%%%%%%%%%%%%%%%%%%%%%%%%%%%%%%%%%%%%%%%%%%%%%%%%%%%%%%%%%%%%%%%%%%%%%%%%%%


%%%%%%%%%%%%%%%%%%%%%%%%%%%% Literaturverzeichnis %%%%%%%%%%%%%%%%%%%%%%%%%%%%%%
%% Bei Fehlern während der Verarbeitung bitte in settings/header/header.tex
%% bei der Einbindung des Pakets biblatex (ungefähr ab Zeile 110,
%% einmal für jede Sprache), biber in bibtex ändern.
\newcommand{\ladeliteratur}{
    \addbibresource{../config/bibliographie.bib}
}

% \splitbibliographytrue       % auskommentieren oder ändern zu \splitbibliographyfalse, wenn Literaturverzeichnis nicht geteilt werden soll

%% Zitierstil
%% mögliche Werte z.B für dhbw: chicago-authordate, mla, apa ; sonstige: ieee, numeric-comp, alphabetic, authoryear
\setzezitierstil{ieee}

%% Zitierbackend
%% mögliche Werte: biber, bibtex -> bei Problemen mit dem Literaturverzeichnis
%% jeweiligen anderen Wert probieren
\setzezitierbackend{bibtex}
%%%%%%%%%%%%%%%%%%%%%%%%%%%%%%%%%%%%%%%%%%%%%%%%%%%%%%%%%%%%%%%%%%%%%%%%%%%%%%%%


%%%%%%%%%%%%%%%%%%%%%%%%%%%%%%%%% Layout %%%%%%%%%%%%%%%%%%%%%%%%%%%%%%%%%%%%%%%
%% Verschiedene Schriftarten
% laut nag Warnung: palatino obsolete, use mathpazo, helvet (option scaled=.95), courier instead
\setzeschriftart{lmodern} % palatino oder goudysans, lmodern, libertine

%% Abstand vor Kapitelüberschriften zum oberen Seitenrand
\setzekapitelabstand{10pt}

%% Spaltenabstand
\setzespaltenabstand{10pt}
%% Zeilenabstand innerhalb einer Tabelle
\setzezeilenabstand{1.5}


%% Flags zum Anzeigen der Bilder

% Logo der Firma im gesamten Dokument anzeigen --> Bild mit Dateinamen
% "company" im images ordner ablegen & Format muss gegebenenfalls in 
% content/additionals/editable/images.tex angepasst werden.
% Ist die Option nicht gesetzt, wird das Logo durch den Text in der
% Variable \setzekopfz ersetzt.
\showfirmenlogotrue                 % auskommentieren oder ändern zu \showfirmenlogofalse, falls kein Logo eingefügt werden soll

% Logo der DHBW im gesamten Dokument anzeigen
\showdhbwlogotrue                   % auskommentieren oder ändern zu \showdhbwlogofalse, falls kein Logo eingefügt werden soll

% Unterschrift als Bild anzeigen --> Bild mit Dateinamen "signature" im 
% images ordner ablegen & Format muss gegebenenfalls in content/additionals/
% editable/images.tex angepasst werden.
\showsignaturetrue                  % auskommentieren oder ändern zu \showsignaturefalse, falls keine Unterschrift eingefügt werden soll
%%%%%%%%%%%%%%%%%%%%%%%%%%%%%%%%%%%%%%%%%%%%%%%%%%%%%%%%%%%%%%%%%%%%%%%%%%%%%%%%
