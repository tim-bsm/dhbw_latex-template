%!TEX root = ../../../document/document.tex

%% Programmiersprachen Highlighting (Listings)
\newcommand{\listingsettings}{%
	\lstset{
		language=C++,			% Standardsprache des Quellcodes
		numbers=left,			% Zeilennummern links
		stepnumber=1,			% Jede Zeile nummerieren.
		%numbersep=5pt,			% 5pt Abstand zum Quellcode
		%numberstyle=\tiny,		% Zeichengrösse 'tiny' für die Nummern.
		breaklines=true,		% Zeilen umbrechen wenn notwendig.
		breakautoindent=true,	% Nach dem Zeilenumbruch Zeile einrücken.
		postbreak=\space,		% Bei Leerzeichen umbrechen.
		tabsize=1,				% Tabulatorgrösse 1
		basicstyle=\ttfamily\footnotesize, % Nichtproportionale Schrift, klein für den Quellcode
		showspaces=false,		% Leerzeichen nicht anzeigen.
		showstringspaces=false,	% Leerzeichen auch in Strings ('') nicht anzeigen.
		extendedchars=true,		% Alle Zeichen vom Latin1 Zeichensatz anzeigen.
		captionpos=b,			% sets the caption-position to bottom
		%backgroundcolor=\color{ListingBackground}, % Hintergrundfarbe des Quellcodes setzen.
		xleftmargin=0pt,		% Rand links
		xrightmargin=0.2em,		% Rand rechts
		frame=single,			% Rahmen an
		frameround=ffff,
		rulecolor=\color{darkgray},	% Rahmenfarbe
		%fillcolor=\color{ListingBackground},
		keywordstyle=\color[rgb]{0.133,0.133,0.6},
		commentstyle=\color[rgb]{0.133,0.545,0.133},
		stringstyle=\color[rgb]{0.627,0.126,0.941},
        aboveskip=1.5em,
	}
	
	\ownlstlanguages
}


\makeatletter

% \savecode{<ID>}{<language>}{<caption>}{<fileName>}
\newcommand\savecode[4]{%
    \@namedef{code@#1}{
		{
			% Count the lines in the file to adjust the spacing of the code-block based on the result
			\newread\myread
			\newcount\linecnt
			\openin\myread=../content/nonTexFiles/code/#4
			\@whilesw\unless\ifeof\myread\fi{%
				\advance\linecnt by \@ne
				\readline\myread t\expandafter o\csname line-\number\linecnt\endcsname
			}
			\advance\linecnt by -1 % correct by one
			\newcommand{\amndig}{\fpeval{ ceil( ln(\linecnt+1)/ln(10) ) }} % get amount of digits in number of lines

			\lstinputlisting[
				language={#2},
				caption={#3},
				label={code:#1},
				% position line numbers depending on the amount of lines in the file
				xleftmargin=\fpeval{2+\amndig*0.5}em,
				framexleftmargin=\fpeval{1.5+\amndig*0.5}em,
			]{../content/nonTexFiles/code/#4}
		}
    }
}

% \usecode{<ID>}
\newcommand\usecode[1]{%
    \@nameuse{code@#1}
}

\makeatother


%!TEX root = ../../../document/document.tex

% FORMAT: \savecode{<ID>}{<language>}{<caption>}{<fileName>}
% For supported languages see: https://en.wikibooks.org/wiki/LaTeX/Source_Code_Listings

\savecode{Beispiel_Code-2}{C++}{Beispiel: indirektes einfügen von Code über externe Datei}{cpp/hello-world.cpp}


% Eigene Sprachen
\newcommand{\ownlstlanguages}{
	\lstdefinelanguage{ini}{
		basicstyle=\ttfamily\small,
		columns=fullflexible,
		morecomment=[s][\color{Orchid}\bfseries]{[}{]},
		morecomment=[l]{\#},
		morecomment=[l]{;},
		commentstyle=\color{gray}\ttfamily,
		morekeywords={},
		otherkeywords={=},
		otherkeywords={:},
		keywordstyle={\color{red}\bfseries}
	}
}