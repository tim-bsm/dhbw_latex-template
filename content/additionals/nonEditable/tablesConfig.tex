%!TEX root = ../../../document/document.tex

\makeatletter

% \savetable{<ID>}{<tableSetting(e.g. !ht)>}{<tabularSetting(e.g. |c|c|c|c|c|c|)}{<description>}{<content>}
\newcommand\savetable[5]{%
    \@namedef{tab:tableSetting@#1}{#2}%
    \@namedef{tab:tabularSetting@#1}{#3}%
    \@namedef{tab:desc@#1}{#4}%
    \@namedef{tab:content@#1}{#5}%
}
\newcommand\tablerow[1]{%
    #1\\
    \hline
}

% \usetable{<ID>}
\newcommand\usetable[1]{%s
    \begin{table}[{\@nameuse{tab:tableSetting@#1}}]
        \centering
        \begin{tabular}[c]{\@nameuse{tab:tabularSetting@#1}}
            \hline
            \@nameuse{tab:content@#1}
            %     \rowcolor{lightgray} 
            %     \textbf{Spalte 1} & 
            %     \textbf{Spalte 2 (\%)} &  
            %     \multicolumn{2}{|c|}{\textbf{Spalte 3}} & 
            %     \multicolumn{2}{|c|}{\textbf{Spalte 4}}\\
            % \hline
            %     A &     10 &    111 &   0,1 &   Dies &      100 \\
            % \hline
            %     B &     20 &    222 &   0,2 &   ist &       200 \\
            % \hline
            %     C &     30 &    333 &   0,3 &   ein &       300 \\
            % \hline
            %     D &     40 &    444 &   0,4 &   Beispiel &  400 \\
            % \hline
            %     \textbf{Summe} &
            %     \textbf{100} &
            %     \textbf{1110} &
            %     \textbf{1} &  &
            %     \textbf{1000}\\
            % \hline
            %     \multicolumn{4}{|c|}{\textbf{Letzte}} & 
            %     \multicolumn{2}{|c|}{\textbf{Reihe}} \\
            % \hline
            %     \makecell{Multilined \\ cell text} & 1 & 2 & 3 & 4 & 5 \\
            % \hline
        \end{tabular}
        \caption{\@nameuse{tab:desc@#1}}
        \label{tab:#1}
    \end{table}
}

\makeatother

%!TEX root = ../../document/document.tex


%%%%%%%%%% Settings %%%%%%%%%%

% Change spaces of tables
\setlength{\tabcolsep}{10pt}
\renewcommand{\arraystretch}{1.5}


%%%%%%%%%% Create new save command %%%%%%%%%%

\makeatletter

% \savetable{<ID>}{<adjustSettings(e.g. center)>}{<tabularSetting(e.g. |c|c|c|c|c|c|)>}{<description>}{<content>}
\newcommand\savetable[5]{%
    \@namedef{tab@#1}{
        \begin{adjustbox}{#2, captionbelow={#4\label{tab:#1}}, nofloat=table}
            \begin{tabular}[c]{#3}
                \hline
                #5
            \end{tabular}
        \end{adjustbox}
    }
}

% \usetable{<ID>}
\newcommand\usetable[1]{%
    \@nameuse{tab@#1}
}

% Adds a shortcut to create a new row in a table
\newcommand\tablerow[1]{%
    #1\\
    \hline
}

\makeatother


%%%%%%%%%% Content %%%%%%%%%%

% !TEX root = ../../document/document.tex
% LTeX: enabled=false


% \savetable{<ID>}{<adjust_settings>}{<column_setting>}{<description>}{<content>}

% adjust_settings: any settings for \begin{adjustbox} e.g. center
% column_setting: any column settings for \begin{tabular} e.g. |c|c|c|c|c|c|
% description: caption to show below the table
% content: the content of the table


\savetable{bsp_normal}{rotate=0, center}{|c|c|c|c|c|c|}{Beispieltabelle}{
    \tablerow{\rowcolor{lightgray}
        \textbf{Spalte 1} &
        \textbf{Spalte 2 (\%)} &
        \multicolumn{2}{|c|}{\textbf{Spalte 3}} & 
        \multicolumn{2}{|c|}{\textbf{Spalte 4}}
    }
    \tablerow{A & 10 & 111 & 0,1 & Dies & 100}
    \tablerow{B & 20 & 222 & 0,2 & ist & 200}
    \tablerow{C & 30 & 333 & 0,3 & ein & 300}
    \tablerow{D & 40 & 444 & 0,4 & Beispiel & 400}
    \tablerow{
        \textbf{Summe} &
        \textbf{100} &
        \textbf{1110} &
        \textbf{1} & 
         &
        \textbf{1000}}
    \tablerow{
        \multicolumn{4}{|c|}{\textbf{Letzte}} &
        \multicolumn{2}{|c|}{\textbf{Reihe}}
    }
    \tablerow{\makecell{Multilined \\ cell text} & 1 & 2 & 3 & 4 & 5}
}

\savetable{bsp_rotiert}{rotate=90, center}{|c|c|c|}{Rotierte Tabelle}{
    \tablerow{A & 10 & test}
    \tablerow{B & 20 & test}
}

