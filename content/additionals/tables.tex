%!TEX root = ../../document/document.tex

% \savetable{<ID>}{<adjust_settings>}{<column_setting>}{<description>}{<content>}

% adjust_settings: any settings for \begin{adjustbox} e.g. center
% column_setting: any column settings for \begin{tabular} e.g. |c|c|c|c|c|c|
% description: caption to show below the table
% content: the content of the table


\savetable{bsp_normal}{rotate=0, center}{|c|c|c|c|c|c|}{Beispieltabelle}{
    \tablerow{\rowcolor{lightgray}
        \textbf{Spalte 1} &
        \textbf{Spalte 2 (\%)} &
        \multicolumn{2}{|c|}{\textbf{Spalte 3}} & 
        \multicolumn{2}{|c|}{\textbf{Spalte 4}}
    }
    \tablerow{A & 10 & 111 & 0,1 & Dies & 100}
    \tablerow{B & 20 & 222 & 0,2 & ist & 200}
    \tablerow{C & 30 & 333 & 0,3 & ein & 300}
    \tablerow{D & 40 & 444 & 0,4 & Beispiel & 400}
    \tablerow{
        \textbf{Summe} &
        \textbf{100} &
        \textbf{1110} &
        \textbf{1} & 
         &
        \textbf{1000}}
    \tablerow{
        \multicolumn{4}{|c|}{\textbf{Letzte}} &
        \multicolumn{2}{|c|}{\textbf{Reihe}}
    }
    \tablerow{\makecell{Multilined \\ cell text} & 1 & 2 & 3 & 4 & 5}
}

\savetable{bsp_rotiert}{rotate=90, center}{|c|c|c|}{Rotierte Tabelle}{
    \tablerow{A & 10 & test}
    \tablerow{B & 20 & test}
}
