% !TEX root = ../../document/document.tex
% LTeX: enabled=false


% \savetable{<ID>}{<adjust_settings>}{<column_setting>}{<description>}{<content>}

% adjust_settings: any settings for \begin{adjustbox} e.g. center
% column_setting: any column settings for \begin{tabular} e.g. |c|c|c|c|c|c|
% description: caption to show below the table
% content: the content of the table


% Use a pipe | to create a vertical line with the default line width. 
% Use !{\vrule width 3pt} instead, to create a vertical line with a custom line width.
\savetable{bsp_normal}{rotate=0, center}{|c!{\vrule width 3pt}c|c|c|c|c|}{Beispieltabelle}{
    \rowcolor{lightgray}
        \textbf{Spalte 1} &
        \textbf{Spalte 2 (\%)} &
        \multicolumn{2}{|c|}{\textbf{Spalte 3}} & 
        \multicolumn{2}{|c|}{\textbf{Spalte 4}}
    % Use \hline to create a horizontal line with the default line height.
    % Use \noalign{\hrule height 3pt} to create a horizontal line with a custom line height.
    \\\noalign{\hrule height 3pt}
    A & 10 & 111 & 0,1 & Dies & 100
    \\\hline
    B & 20 & 222 & 0,2 & ist & 200
    \\\hline
    \multirow{2}{*}{C} & 30 & 333 & 0,3 & ein & 300
    \\\cline{2-6}
     & 40 & 444 & 0,4 & Beispiel & 400
    \\\hline
        \textbf{Summe} &
        \textbf{100} &
        \textbf{1110} &
        \textbf{1} & 
         &
        \textbf{1000}
    \\\hline
        \multicolumn{4}{|c|}{\textbf{Letzte}} &
        \multicolumn{2}{|c|}{\textbf{Reihe}}
    \\\hline
    \makecell{Multilined \\ cell text} & 1 & 2 & 3 & 4 & 5
    \\\hline
}

\savetable{bsp_rotiert}{rotate=90, center}{|c|c|c|}{Rotierte Tabelle}{
    A & 10 & test
    \\\hline
    B & 20 & test
    \\\hline
}
