%!TEX root = ../../../document/document.tex

% Nur verwendete Akronyme werden letztlich im Abkürzungsverzeichnis des Dokuments angezeigt
% Verwendung: 
% \acro{BSP}{Board Support Package} % Beispielabkürzung
% 		\ac{Abk.}   --> fügt die Abkürzung ein, beim ersten Aufruf wird zusätzlich automatisch die ausgeschriebene Version davor eingefügt bzw. in einer Fußnote (hierfür muss in header.tex \usepackage[printonlyused,footnote]{acronym} stehen) dargestellt
% 		\acf{Abk.}   --> fügt die Abkürzung UND die Erklärung ein
% 		\acl{Abk.}   --> fügt nur die Erklärung ein
% 		\acp{Abk.}  --> gibt Plural aus (angefügtes 's'); das zusätzliche 'p' funktioniert auch bei obigen Befehlen
% 		\acs{Abk.}   -->  fügt die Abkürzung ein
% für weitere Optionen siehe: http://www.namsu.de/Extra/pakete/Acronym.pdf


%A:

%B:

%C:

%D:
\acro{dta}{das Test-Acronym}
\acro{DHBW}{Duale Hochschule Baden-Württemberg}

%E:

%F:

%G:

%H:

%I:

%J:

%K:

%L:

%M:

%N:

%O:

%P:

%Q:

%R:

%S:

%T:

%U:

%V:

%W:

%X:

%Y:

%Z:
