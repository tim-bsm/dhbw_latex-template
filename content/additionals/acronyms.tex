% !TEX root = ../../document/document.tex
% LTeX: enabled=false

% Only used acronyms will be displayed in the list of acronyms.
% When defining the acronyms with their, use \- between each
% syllable for LaTeX to cut them off correctly at the end of each line.
% For more options see: http://www.namsu.de/Extra/pakete/Acronym.pdf

% Usage:
% \acro{BSP}{Board Support Package} % Example acronym
% \ac{Abbr.}   --> inserts the abbreviation, at the first call the full version is inserted
% \acf{Abbr.}   --> inserts the abbreviation AND the explanation
% \acl{Abbr.}   --> inserts only the explanation
% \acp{Abbr.}  --> inserts the plural form (adds 's'); the additional 'p' works with the above commands as well
% \acs{Abbr.}   --> inserts the abbreviation

% If the space between the acronym and the explanation is too small, you can adjust it by going into the config/additionals/acronyms.tex file and changing the 'ABCDEFGHIJ' to the longest acronym you have. 


%A:

%B:

%C:

%D:
\acro{dta}{das Test-\-Ac\-ro\-nym}
\acro{DHBW}{Du\-a\-le Hoch\-schu\-le Ba\-den-\-Würt\-tem\-berg}

%E:

%F:

%G:

%H:

%I:

%J:

%K:

%L:

%M:

%N:

%O:

%P:

%Q:

%R:

%S:

%T:

%U:

%V:

%W:

%X:

%Y:

%Z:
