%!TEX root = ../../document/document.tex

% For every chapter create a file in the /content/chapters folder with the name
% chapter-XX.tex where XX is the chapter number (e.g. 01, 02, 03, ..., 99).


\chapter{Beispiele zur Verwendung von LaTeX}
   
\section{Unterkapitel, Zitate, Referenzen, Formeln und Listings}\label{sec:Unterkapitel}

\subsection{Unterunterkapitel}

\enquote{Dies ist ein Beispiel für die Zitierfunktion} \cite[Vgl.][S. 1 ff.]{2024}. Zitate werden in der Reihenfolge ihres Auftretens im Text nummeriert. Nicht verwendete Zitate werden im Literaturverzeichnis unter \enquote{Weitere Quellen} aufgeführt.

% Man kann ebenfalls einen \gls{Glossareintrag} oder mehrere \glspl{Glossareintrag} verwenden.

Hier ein Beispiel für eine Formel (\autoref{eq:OhmschesGesetz}):
\useequation{OhmschesGesetz}          % Ausgelagerte Formel einfügen

Hier zwei Beispiele für Code (\autoref{code:Beispiel_Code-1} und \autoref{code:Beispiel_Code-2}):

% Code im .tex file --> nicht empfohlen, u.a., da Tabs mitverwendet werden und
% das .tex file dadurch unübersichtlich wird (s. listing 1.1 im PDF-File)
\begin{lstlisting}[language=C++, caption={Beispiel: direktes einfügen von Code}, label={code:Beispiel_Code-1}]
    #include <iostream>

    int main() {
        std::cout << "Hello World!" << std::endl;
        return 0;
    }
\end{lstlisting}

% Code in einer externen Datei
\usecode{Beispiel_Code-2}                % Ausgelagerten Code einfügen

%%%%%%%%%%%%%%%%%%%%%%%%%% NEUE SEITE %%%%%%%%%%%%%%%%%%%%%%%%%%

\newpage
\section{Bilder, Tabellen und Listen}

\useimage{bsp}

Wie in \autoref{img:bsp} zu sehen, ist das Logo der \ac{DHBW} ein sehr schönes Logo. Das $\lambda$ der \ac{lambda} ist ebenfalls ein sehr schönes Zeichen. Des Weiteren kann auf die Kapitelnummer referenziert werden: \autoref{sec:Unterkapitel} und mit \hyperref[sec:Unterkapitel]{Test Unterkapitel} können eigene Texte für die Referenzen vergeben werden. Mit \todo{Aufgabe für mich} können Aufgaben markiert werden, die noch erledigt werden müssen.\cite{A01}


\vspace{0.5cm}
\usetable{bsp_normal}

Auch hier kann wieder auf die Tabellen referenziert werden: \autoref{tab:bsp_normal}, \autoref{tab:bsp_rotiert}.


\usetable{bsp_rotiert}

Hier ein Beispiel für eine Liste:
\begin{itemize}[label=\textbullet]
    \item Punkt 1
    \item Punkt 2
    \item Punkt 3
\end{itemize}

\Acl{dta} wird großgeschrieben und \ac{dta} klein, wie es definiert wurde. 
