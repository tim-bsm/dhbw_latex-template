%!TEX root = ../../document/document.tex

% Zum erstellen verschiedener Kapitel, jeweils eine Datei pro Kapitel erstellen
% und in content/chapters/XXChapter.tex (XX = Kapitelnummer -> 01, 02, 03, ...,
% 99) einbinden. 


\chapter{Beispiele zur Verwendung von LaTeX}
   
\section{Unterkapitel, Zitate, Referenzen und Formeln}
\label{sec:Unterkapitel}

\subsection{Unterunterkapitel}

Dies ist ein Beispiel für die Zitierfunktion. \cite[Vgl.][S. 1 ff.]{Mustermann2023} 


\vspace{0.5cm}
Hier ein Beispiel für eine Formel

\begin{equation}
    \formelentry{Ohm'sches Gesetz}      % Zu Formelverzeichnis hinzufügen
    {R} = \frac{U}{I}
    \label{eq:OhmschesGesetz}
\end{equation}
\begin{center}
    Formel \ref{eq:OhmschesGesetz}: Ohm'sches Gesetz
\end{center}



\vspace{0.5cm}
Hier zwei Beispiele für Code (\ref{lst:Beispiel_Code-1} und \ref{lst:Beispiel_Code-2}):

% Code im .tex file --> nicht empfohlen, u.a., da Tabs mitverwendet werden und
% das .tex file dadurch unübersichtlich wird (s. listing 1.1 im PDF-File)
\begin{lstlisting}[language=C++, caption={Beispiel: direktes einfügen von Code}, label={lst:Beispiel_Code-1}]
    #include <iostream>

    int main() {
        std::cout << "Hello World!" << std::endl;
        return 0;
    }
\end{lstlisting}

% Code in einer externen Datei
\lstinputlisting[language=C++, caption={Beispiel: indirektes einfügen von Code über externe Datei}, label={lst:Beispiel_Code-2}]
{../content/nonTexFiles/code/hello-world.cpp}

%%%%%%%%%%%%%%%%%%%%%%%%%% NEUE SEITE %%%%%%%%%%%%%%%%%%%%%%%%%%

\newpage
\section{Bilder, Tabellen und Listen}


\begin{figure}[h]
    \centering
    \includegraphics[width=0.7\linewidth]{dhbw_de.png}
    \caption{Beispiel: einfügen eines Bildes \cite{Mustermann2023}}
    \label{fig:Bsp_Bild}
\end{figure}

Wie in \autoref{fig:Bsp_Bild} zu sehen, ist das Logo der \ac{DHBW} ein sehr schönes Logo. Das $\lambda$ der \ac{lambda} ist ebenfalls ein sehr schönes Zeichen. Des weiteren kann auf die Kapitelnummer referenziert werden: \autoref{sec:Unterkapitel}.


\vspace{0.5cm}
\begin{table}[h]
    \centering
    \begin{tabular}[c]{|c|c|c|c|c|c|}
        \hline
            \textbf{Spalte 1} & 
            \textbf{Spalte 2 (\%)} &  
            \multicolumn{2}{|c|}{\textbf{Spalte 3}} & 
            \multicolumn{2}{|c|}{\textbf{Spalte 4}}\\
        \hline
            A &     10 &    111 &   0,1 &   Dies &      100 \\
        \hline
            B &     20 &    222 &   0,2 &   ist &       200 \\
        \hline
            C &     30 &    333 &   0,3 &   ein &       300 \\
        \hline
            D &     40 &    444 &   0,4 &   Beispiel &  400 \\
        \hline
            \textbf{Summe} &
            \textbf{100} &
            \textbf{1110} &
            \textbf{1} &  &
            \textbf{1000}\\
        \hline
            \multicolumn{4}{|c|}{\textbf{Letzte}} & 
            \multicolumn{2}{|c|}{\textbf{Reihe}} \\
        \hline
    \end{tabular}
    \caption{Beispiel: einfügen einer Tabelle}
    \label{tab:Bsp_Tabelle}
\end{table}

Auch hier kann wieder auf die Tabelle referenziert werden: \ref{tab:Bsp_Tabelle}.


\vspace{0.5cm}
Hier ein Beispiel für eine Liste:
\begin{itemize}[label=\textbullet]
    \item Punkt 1
    \item Punkt 2
    \item Punkt 3
\end{itemize}