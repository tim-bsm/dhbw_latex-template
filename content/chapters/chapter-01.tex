%!TEX root = ../../document/document.tex

% Zum erstellen verschiedener Kapitel, jeweils eine Datei pro Kapitel erstellen
% und in content/written/chapters/chapter-XX.tex (XX = Kapitelnummer -> 01, 02, 03,
% ..., 99) einbinden. 


\chapter{Beispiele zur Verwendung von LaTeX}
   
\section{Unterkapitel, Zitate, Referenzen, Formeln und Listings}
\label{sec:Unterkapitel}

\subsection{Unterunterkapitel}

\enquote{Dies ist ein Beispiel für die Zitierfunktion} \cite[Vgl.][S. 1 ff.]{2024}. 


\vspace{0.5cm}
Hier ein Beispiel für eine Formel (\ref{eq:OhmschesGesetz}):
\useequation{OhmschesGesetz}          % Ausgelagerte Formel einfügen

\vspace{0.5cm}
Hier zwei Beispiele für Code (\ref{lst:Beispiel_Code-1} und \ref{lst:Beispiel_Code-2}):

% Code im .tex file --> nicht empfohlen, u.a., da Tabs mitverwendet werden und
% das .tex file dadurch unübersichtlich wird (s. listing 1.1 im PDF-File)
\begin{lstlisting}[language=C++, caption={Beispiel: direktes einfügen von Code}, label={lst:Beispiel_Code-1}]
    #include <iostream>

    int main() {
        std::cout << "Hello World!" << std::endl;
        return 0;
    }
\end{lstlisting}

% Code in einer externen Datei
\uselisting{Beispiel_Code-2}                % Ausgelagerten Code einfügen

%%%%%%%%%%%%%%%%%%%%%%%%%% NEUE SEITE %%%%%%%%%%%%%%%%%%%%%%%%%%

\newpage
\section{Bilder, Tabellen und Listen}

\useimage{bsp}

Wie in \autoref{img:bsp} zu sehen, ist das Logo der \ac{DHBW} ein sehr schönes Logo. Das $\lambda$ der \ac{lambda} ist ebenfalls ein sehr schönes Zeichen. Des Weiteren kann auf die Kapitelnummer referenziert werden: \autoref{sec:Unterkapitel} und mit \hyperref[sec:Unterkapitel]{Test Unterkapitel} können eigene Texte für die Referenzen vergeben werden. Mit \todo{Aufgabe für mich} können Absätze markiert werden.


\vspace{0.5cm}
\usetable{bsp_normal}

Auch hier kann wieder auf die Tabellen referenziert werden: \autoref{tab:bsp_normal}, \autoref{tab:bsp_rotiert}.


\newpage
\usetable{bsp_rotiert}

Hier ein Beispiel für eine Liste:
\begin{itemize}[label=\textbullet]
    \item Punkt 1
    \item Punkt 2
    \item Punkt 3
\end{itemize}