%!TEX root = ../document/document.tex

%nur verwendete Akronyme werden letztlich im Abkürzungsverzeichnis des Dokuments angezeigt
%Verwendung: 
%\acro{BSP}{Board Support Package} % Beispielabkürzung
%		\ac{Abk.}   --> fügt die Abkürzung ein, beim ersten Aufruf wird zusätzlich automatisch die ausgeschriebene Version davor eingefügt bzw. in einer Fußnote (hierfür muss in header.tex \usepackage[printonlyused,footnote]{acronym} stehen) dargestellt
%		\acf{Abk.}   --> fügt die Abkürzung UND die Erklärung ein
%		\acl{Abk.}   --> fügt nur die Erklärung ein
%		\acp{Abk.}  --> gibt Plural aus (angefügtes 's'); das zusätzliche 'p' funktioniert auch bei obigen Befehlen
%		\acs{Abk.}   -->  fügt die Abkürzung ein
%	siehe auch: http://golatex.de/wiki/%5Cacronym


%A:
\acro{AC}{Wechselstrom (engl. Alternating current)}
\acro{ACR}{automatisierter Aufladeroboter (engl. Automated Charging Robot)}
\acro{ALaPuN}{Automatisches Ladesystem für PkWs und leichte Nutzfahrzeuge}
\acro{AVP}{Autnonomes Parken im Parkhaus (engl. Automated Valet Parking)}
%B:
\acro{BEV}{Batterieelektrisch angetriebene PkW (engl. Battery electric vehicles}
\acro{BU}{Business Unit}
%C:
\acro{CCS}{Combined Charging System}
\acro{Cobot}{Kollaborativer Roboter (engl. Collaborative Robot}
%D:
\acro{DC}{Gleichstrom (engl. Direct current)}
%E:
\acro{EV}{Elektrofahrzeug (engl. Electric vehicles)}
%F:

%G:

%H:

%I:
\acro{I}{Stromstärke}
%J:

%K:
\acro{KI}{Künstliche Intelligenz}
%L:
\acro{LiDAR}{Light Detection and Ranging}
%M:

%N:
\acro{n}{Anzahl}
\acro{NWA}{Nutzwertanalyse}
%O:

%P:
\acro{PHEV}{Plug-in Hybride (engl. Plug-In Hybrid electric vehicles)}
\acro{PV-Anlage}{Photovoltaik-Anlage}
\acro{P}{Leistung}
%Q:

%R:
\acro{r}{Radius}
%S:

%T:

%U:
\acro{U}{Spannung}
%V:

%W:

%X:

%Y:

%Z:
