%!TEX root = ../../document/document.tex

\RequirePackage[l2tabu, orthodox]{nag}	% weist in Commandozeile bzw. log auf veraltete LaTeX Syntax hin

\documentclass[%
    pdftex,
    oneside,			% Einseitiger Druck.
    12pt,				% Schriftgroesse
    parskip=half,		% Halbe Zeile Abstand zwischen Absätzen.
    %topmargin = 10pt,	% Abstand Seitenrand (Std:1in) zu Kopfzeile [laut log: unused]
    headheight = 33pt,	% Höhe der Kopfzeile
    %headsep = 30pt,	% Abstand zwischen Kopfzeile und Text Body  [laut log: unused]
    headsepline,		% Linie nach Kopfzeile.
    footsepline,		% Linie vor Fusszeile.
    %footheight = 16pt,	% Höhe der Fusszeile
    abstract=true,      % Abstract Überschriften
    DIV=calc,		    % Satzspiegel berechnen
    BCOR=8mm,		    % Bindekorrektur links: 8mm
    headinclude=false,	% Kopfzeile nicht in den Satzspiegel einbeziehen
    footinclude=false,	% Fußzeile nicht in den Satzspiegel einbeziehen
    listof=totoc,		% Abbildungs-/ Tabellenverzeichnis im Inhaltsverzeichnis darstellen
    toc=bibliography,	% Literaturverzeichnis im Inhaltsverzeichnis darstellen
]{scrreprt}	% Koma-Script report-Klasse, fuer laengere Bachelorarbeiten alternativ auch: scrbook



% Einstellungen laden
\usepackage{xstring}
\usepackage{ifpdf}
\usepackage{ifluatex}


\usepackage{lastpage}
\usepackage{fancyhdr}
\newcommand{\einstellung}[1]{%
    \expandafter\newcommand\csname #1\endcsname{}
    \expandafter\newcommand\csname setze#1\endcsname[1]{\expandafter\renewcommand\csname#1\endcsname{##1}}
}
\newcommand{\langstr}[1]{\einstellung{lang#1}}

% Flag für die Selbstständigkeitserklärung, Default: true
\newif\ifselbsterkl
\selbsterklfalse

% Flag für das Wasserzeichen auf dem Deckblatt, default: false
\newif\ifwatermark
\watermarkfalse

% Flaf für das Anzeigen der Vertraulichkeitsstufe, default: false
\newif\ifgreendot
\greendotfalse

% Flag für roten Vertraulichkeitspunkt, default: false
\newif\ifreddot
\reddotfalse

% Flag für gelben Vertraulichkeitspunkt, default: false
\newif\ifyellowdot
\yellowdotfalse

% Flag für das Unterschriftenblatt, default: false
\newif\ifunterschriftenblatt
\unterschriftenblattfalse

% Flag für Einfügen der Seitenzahl bei Verweis auf Kapitel/Abschnitt, default: false
\newif\ifrefWithPages
\refWithPagesfalse

% Flag für Einfügen der Abstracts in deutsch und englisch, default: false
\newif\ifbothabstracts
\bothabstractsfalse

% Flag für Einfügen des Abkürzungsverzeichnis
\newif\ifabkverz
\abkverzfalse

% Flag für Einfügen des Abbildungsverzeichnisses
\newif\ifabbverz
\abbverzfalse

% Flag für Einfügen des Formelverzeichnisses
\newif\ifformelverz
\formelverzfalse

% Flag für Einfügen des Formelgroessenverzeichnisses
\newif\ifformelgroeverz
\formelgroeverzfalse 

% Flag für Einfügen des Listingsverzeichnisses
\newif\iflistverz
\listverzfalse

% Flag für Einfügen des Tabellenverzeichnisses
\newif\iftableverz
\tableverzfalse

% Flag für Einfügen des Sperrvermerks
\newif\ifsperrvermerk
\sperrvermerkfalse

% Flag für Einfügen des Abstracts
\newif\ifabstract
\abstractfalse

% Flag für Anhang
\newif\ifappendix
\appendixfalse

% Flag für Literaturverzeichnis
\newif\ifliteratur
\literaturfalse

% Flag für Glossar
\newif\ifglossar
\glossarfalse

% Flag für Anhang
\newif\ifanhang
\anhangfalse

% Flag für Inhaltsverzeichnis
\newif\ifinhalt
\inhaltfalse

% Flag für Formblatt
\newif\ifformblatt
\formblattfalse

% Flag für Reviewer
\newif\ifreviewer
\reviewerfalse

% Flag für Anzeigen des Firmenlogos; default: false
\newif\ifshowfirmenlogo
\showfirmenlogofalse

% Flag für Anzeigen des DHBW-Logos; default: false
\newif\ifshowdhbwlogo
\showdhbwlogofalse

% Flag für Anzeigen der Unterschrift; default: false
\newif\ifshowsignature
\showsignaturefalse


\newcommand{\createsettings}{
    \einstellung{martrikelnr}
    \einstellung{titel}
    \einstellung{kurs}
    \einstellung{datumAnfang}
    \einstellung{datumAbgabe}
    \einstellung{firma}
    \einstellung{firmenort}
    \einstellung{abgabeort}
    \einstellung{abschluss}
    \einstellung{studiengang}
    \einstellung{dhbw}
    \einstellung{hochschule}
    \einstellung{betreuer}
    \einstellung{gutachter}
    \einstellung{zeitraum}
    \einstellung{arbeit}
    \einstellung{autor}
    \einstellung{sprache}
    \einstellung{schriftart}
    \einstellung{kapitelabstand}
    \einstellung{spaltenabstand}
    \einstellung{zeilenabstand}
    \einstellung{zitierstil}
    \einstellung{zitierbackend}
    \einstellung{selbsterkl}
    \einstellung{semester}
    \einstellung{studienrichtung}
    \einstellung{jahrgang}
    \einstellung{abteilung}
    \einstellung{standort}
    \einstellung{fusszmitte}
    \einstellung{kopfz}
}

% Create flags for some settings accross the document; default set on 'false'
\newcommand{\createifs}{
    % Flag für die Selbstständigkeitserklärung
    \newif\ifselbsterkl
    \selbsterklfalse

    % Flag für das Wasserzeichen auf dem Deckblatt
    \newif\ifwatermark
    \watermarkfalse

    % Flaf für das Anzeigen der Vertraulichkeitsstufe
    \newif\ifgreendot
    \greendotfalse

    % Flag für roten Vertraulichkeitspunkt
    \newif\ifreddot
    \reddotfalse

    % Flag für gelben Vertraulichkeitspunkt
    \newif\ifyellowdot
    \yellowdotfalse

    % Flag für Einfügen der Seitenzahl bei Verweis auf Kapitel/Abschnitt
    \newif\ifrefWithPages
    \refWithPagesfalse

    % Flag für Einfügen des Abkürzungsverzeichnis
    \newif\ifabkverz
    \abkverzfalse

    % Flag für Einfügen des Abbildungsverzeichnisses
    \newif\ifabbverz
    \abbverzfalse

    % Flag für Einfügen des Formelverzeichnisses
    \newif\ifformelverz
    \formelverzfalse

    % Flag für Einfügen des Formelgroessenverzeichnisses
    \newif\ifformelgroeverz
    \formelgroeverzfalse 

    % Flag für Einfügen des Listingsverzeichnisses
    \newif\iflistverz
    \listverzfalse

    % Flag für Einfügen des Tabellenverzeichnisses
    \newif\iftableverz
    \tableverzfalse

    % Flag für Einfügen des Sperrvermerks
    \newif\ifsperrvermerk
    \sperrvermerkfalse

    % Flag für Einfügen des deutschen Abstracts
    \newif\ifabstractde
    \abstractdefalse

    % Flag für Einfügen des englischen Abstracts
    \newif\ifabstracten
    \abstractenfalse

    % Flag für Anhang
    \newif\ifappendix
    \appendixfalse

    % Flag für Literaturverzeichnis
    \newif\ifliteratur
    \literaturfalse

    % Flag für Glossar
    \newif\ifglossar
    \glossarfalse

    % Flag für Anhang
    \newif\ifanhang
    \anhangfalse

    % Flag für Inhaltsverzeichnis
    \newif\ifinhalt
    \inhaltfalse

    % Flag für Reviewer
    \newif\ifreviewer
    \reviewerfalse

    % Flag für Anzeigen des Firmenlogos
    \newif\ifshowfirmenlogo
    \showfirmenlogofalse

    % Flag für Anzeigen des DHBW-Logos
    \newif\ifshowdhbwlogo
    \showdhbwlogofalse

    % Flag für Anzeigen der Unterschrift
    \newif\ifshowsignature
    \showsignaturefalse

    % Flag zum Trennen des Literaturverzeichnisses
    \newif\ifsplitbibliography
    \splitbibliographyfalse
}

\createsettings
\createifs %# TODO move config of additionals in config  % verfügbare Einstellungen
%%%%%%%%%%%%%%%%%%%%%%%%%%%%%%%%%%%%%%%%%%%%%%%%%%%%%%%%%%%%%%%%%%%%%%%%%%%%%%%
%                                   Einstellungen
%
% Hier können alle relevanten Einstellungen für diese Arbeit gesetzt werden.
% Dazu gehören Angaben u.a. über den Autor sowie Formatierungen.
%
%
%%%%%%%%%%%%%%%%%%%%%%%%%%%%%%%%%%%%%%%%%%%%%%%%%%%%%%%%%%%%%%%%%%%%%%%%%%%%%%%


%%%%%%%%%%%%%%%%%%%%%%%%%%%%%%%%%%%% Sprache %%%%%%%%%%%%%%%%%%%%%%%%%%%%%%%%%%
%% Aktuell sind Deutsch und Englisch unterstützt.
%% Es werden nicht nur alle vom Dokument erzeugten Texte in
%% der entsprechenden Sprache angezeigt, sondern auch weitere
%% Aspekte angepasst, wie z.B. die Anführungszeichen und
%% Datumsformate.
\setzesprache{de} % de oder en
%%%%%%%%%%%%%%%%%%%%%%%%%%%%%%%%%%%%%%%%%%%%%%%%%%%%%%%%%%%%%%%%%%%%%%%%%%%%%%%


%%%%%%%%%%%%%%%%%%%%%%%%%%%%%%%%%%% Angaben  %%%%%%%%%%%%%%%%%%%%%%%%%%%%%%%%%%
%% Die meisten der folgenden Daten werden auf dem
%% Deckblatt angezeigt, einige auch im weiteren Verlauf
%% des Dokuments.
\setzemartrikelnr{1234567}
\setzekurs{TINF22ITA}
\setzetitel{Thema der Arbeit}
\setzedatumAnfang{01.01.2023}
\setzedatumAbgabe{31.12.2023}
\setzefirma{MAHLE International GmbH}
\setzefirmenort{Stuttgart}
\setzeabgabeort{Stuttgart}
\setzeabschluss{Bachelor of Science}
\setzestudiengang{IT-Automotive}
\setzedhbw{Stuttgart}
\setzehochschule{Duale Hochschule Baden-Württemberg}
\setzebetreuer{B. Sc. Max Mustermann}
\setzegutachter{Prof. Dr. rer. nat. Gustaf Gutachter}
\setzezeitraum{24.04.2023 - 14.08.2023}
\setzearbeit{T3\_1000}
\setzeautor{John Doe}
\setzesemester{4.}
\setzestudienrichtung{Informatik}
\setzejahrgang{2022}
\setzeabteilung{Abteilung}
\setzestandort{Stuttgart}
\setzefusszmitte{\arbeit}

\inhalttrue                 % auskommentieren oder ändern zu \inhaltfalse, falls kein Inhaltsverzeichnis eingefügt werden soll
\unterschriftenblatttrue    % auskommentieren oder ändern zu \unterschriftenblattfalse, falls kein Unterschriftenblatt eingefügt werden soll
\selbsterkltrue             % auskommentieren oder ändern zu \selbsterklfalse, wenn keine Selbstständigkeitserklärung benötigt wird
\sperrvermerktrue           % auskommentieren oder ändern zu \sperrvermerkfalse, wenn kein Sperrvermerk benötigt wird
\abkverztrue                % auskommentieren oder ändern zu \abkverzfalse, wenn kein Abkürzungsverzeichnis benötigt wird
\abbverztrue                % auskommentieren oder ändern zu \abbverzfalse, wenn kein Abbildungsverzeichnis benötigt wird
\tableverztrue              % auskommentieren oder ändern zu \tableverzfalse, wenn kein Tabellenverzeichnis benötigt wird
\listverztrue               % auskommentieren oder ändern zu \listverzfalse, wenn kein Listingsverzeichnis benötigt wird
\formelverztrue             % auskommentieren oder ändern zu \formelverzfalse, wenn kein Formelverzeichnis benötigt wird
\formelgroeverztrue			% auskommentieren oder ändern zu \formelgroeverzfalse, wenn kein Formelgrößenverzeichnis benötigt wird
\abstracttrue               % auskommentieren oder ändern zu \abstractfalse, wenn kein Abstract gewünscht ist
\bothabstractstrue          % auskommentieren oder ändern zu \bothabstractsfalse, wenn nur der Abstract in der Hauptsprache eingefügt werden soll
\appendixtrue               % auskommentieren oder ändern zu \appendixfalse, wenn kein Anhang gewünscht ist
\literaturtrue              % auskommentieren oder ändern zu \literaturfalse, wenn kein Literaturverzeichnis gewünscht ist (\appendixtrue muss gesetzt sein!)
\glossartrue                % auskommentieren oder ändern zu \glossarfalse, wenn kein Glossar gewünscht ist (\appendixtrue muss gesetzt sein!)
\anhangtrue                 % auskommentieren oder ändern zu \anhangfalse, wenn kein Anhang gewünscht ist
%\formblatttrue              % auskommentieren oder ändern zu \formblattfalse, wenn kein Formblatt gewünscht ist

%\refWithPagestrue           % ändern zu \refWithPagestrue, wenn die Seitenzahl bei Verweisen auf Kapitel engefügt werden sollen

\reviewertrue				% auskommentiren oder ändern zu \reviewerfalse wenn kein Gutachter gesetzt werden muss

% Angabe des roten/gelben/grünen Punktes auf dem Titelblatt zur Kennzeichnung der Vertraulichkeitsstufe.
% Mögliche Angaben sind \greendottrue, \yellowdottrue und \reddottrue. Werden mehrere angegeben, wird der Punkt mit der höheren
% Vertraulichkeitsstufe angezeigt.
% Wird keines der Kommandos angegeben, wird kein Punkt gezeichnet.
%\greendottrue
%\yellowdottrue
%\reddottrue

%\watermarktrue              % auskommentieren oder ändern zu \watermarkfalse, wenn kein "Vertraulich"-Wasserzeichen auf den Seiten eingefügt werden soll
%%%%%%%%%%%%%%%%%%%%%%%%%%%%%%%%%%%%%%%%%%%%%%%%%%%%%%%%%%%%%%%%%%%%%%%%%%%%%%%%


%%%%%%%%%%%%%%%%%%%%%%%%%%%% Literaturverzeichnis %%%%%%%%%%%%%%%%%%%%%%%%%%%%%%
%% Bei Fehlern während der Verarbeitung bitte in settings/header/header.tex
%% bei der Einbindung des Pakets biblatex (ungefähr ab Zeile 110,
%% einmal für jede Sprache), biber in bibtex ändern.
\newcommand{\ladeliteratur}{
    \addbibresource{../settings/bibliographie.bib}
}

%% Zitierstil
%% siehe: http://ctan.mirrorcatalogs.com/macros/latex/contrib/biblatex/doc/biblatex.pdf (3.3.1 Citation Styles)
%% mögliche Werte z.B numeric-comp, alphabetic, authoryear
\setzezitierstil{numeric-comp}
%%%%%%%%%%%%%%%%%%%%%%%%%%%%%%%%%%%%%%%%%%%%%%%%%%%%%%%%%%%%%%%%%%%%%%%%%%%%%%%%


%%%%%%%%%%%%%%%%%%%%%%%%%%%%%%%%% Layout %%%%%%%%%%%%%%%%%%%%%%%%%%%%%%%%%%%%%%%
%% Verschiedene Schriftarten
% laut nag Warnung: palatino obsolete, use mathpazo, helvet (option scaled=.95), courier instead
\setzeschriftart{lmodern} % palatino oder goudysans, lmodern, libertine

%% Abstand vor Kapitelüberschriften zum oberen Seitenrand
\setzekapitelabstand{10pt}

%% Spaltenabstand
\setzespaltenabstand{10pt}
%% Zeilenabstand innerhalb einer Tabelle
\setzezeilenabstand{1.5}


%% Flags zum Anzeigen der Bilder

% Logo der Firma im gesamten Dokument anzeigen --> Bild mit Dateinamen
% "company" im images ordner ablegen & Format muss gegebenenfalls in 
% content/additionals/cover-sheet.tex und settings/config/config.tex
% unter "Kopf-/Fußzeilenwechsel" (ca. Z. 200) angepasst werden.
% Ist die Option nicht gesetzt, wird das Logo durch den Text in der Variable
% \setzelangkopfz in settings/lang/de.tex bzw. settings/lang/en.tex ersetzt.
\showfirmenlogotrue                 % auskommentieren oder ändern zu \showfirmenlogofalse, falls kein Logo eingefügt werden soll

% Logo der DHBW im gesamten Dokument anzeigen
\showdhbwlogotrue                   % auskommentieren oder ändern zu \showdhbwlogofalse, falls kein Logo eingefügt werden soll

% Unterschrift als Bild anzeigen --> Bild mit Dateinamen "signature" im 
% images ordner ablegen & Format muss gegebenenfalls in content/additionals/
% declaration.tex und content/additionals/confidentiality-statement.tex
% angepasst werden
\showsignaturetrue                  % auskommentieren oder ändern zu \showsignaturefalse, falls keine Unterschrift eingefügt werden soll
%%%%%%%%%%%%%%%%%%%%%%%%%%%%%%%%%%%%%%%%%%%%%%%%%%%%%%%%%%%%%%%%%%%%%%%%%%%%%%%% % lese Einstellungen

\newcommand{\iflang}[2]{%
  \IfStrEq{\sprache}{#1}{#2}{}
}

\langstr{abkverz}
\langstr{anhang}
\langstr{glossar}
\langstr{artikelstudiengang}
\langstr{studiengang}
\langstr{anderdh}
\langstr{von}
\langstr{dbbearbeitungszeit}
\langstr{dbmatriknr}
\langstr{dbkurs}
\langstr{dbfirma}
\langstr{dbbetreuer}
\langstr{dbgutachter}
\langstr{sperrvermerk}
\langstr{abstract}
\langstr{listingname}
\langstr{listlistingname}
\langstr{listingautorefname}
\langstr{selbsterkl}
\langstr{formelsammlung}
\langstr{seite}
\langstr{seitevon}
\langstr{stand}
\langstr{formelgroeverz}
\langstr{formel}
\langstr{literatur}
\langstr{weblinks}
\langstr{weiterequellen}
 % verfügbare Strings
\input{../settings/lang/\sprache.tex} % Übersetzung einlesen


%\lstset{language=Matlab}
\newcommand{\citem}[1]{\item[\texttt{#1}]} % Code-Item für description-Liste
\newcommand\todo[1]{\textit{\textcolor{red}{TODO: #1}}} % Todo-Item


%% Farben (Angabe in HTML-Notation mit großen Buchstaben)
\newcommand{\ladefarben}{%
	\definecolor{LinkColor}{HTML}{00007A}
	\definecolor{ListingBackground}{HTML}{FCF7DE}
}


%% Programmiersprachen Highlighting (Listings)
\newcommand{\listingsettings}{%
	\lstset{
		language=C++,			% Standardsprache des Quellcodes
		numbers=left,			% Zeilennummern links
		stepnumber=1,			% Jede Zeile nummerieren.
		%numbersep=5pt,			% 5pt Abstand zum Quellcode
		%numberstyle=\tiny,		% Zeichengrösse 'tiny' für die Nummern.
		breaklines=true,		% Zeilen umbrechen wenn notwendig.
		breakautoindent=true,	% Nach dem Zeilenumbruch Zeile einrücken.
		postbreak=\space,		% Bei Leerzeichen umbrechen.
		tabsize=1,				% Tabulatorgrösse 1
		basicstyle=\ttfamily\footnotesize, % Nichtproportionale Schrift, klein für den Quellcode
		showspaces=false,		% Leerzeichen nicht anzeigen.
		showstringspaces=false,	% Leerzeichen auch in Strings ('') nicht anzeigen.
		extendedchars=true,		% Alle Zeichen vom Latin1 Zeichensatz anzeigen.
		captionpos=b,			% sets the caption-position to bottom
		%backgroundcolor=\color{ListingBackground}, % Hintergrundfarbe des Quellcodes setzen.
		xleftmargin=0pt,		% Rand links
		xrightmargin=0pt,		% Rand rechts
		frame=single,			% Rahmen an
		frameround=ffff,
		rulecolor=\color{darkgray},	% Rahmenfarbe
		%fillcolor=\color{ListingBackground},
		keywordstyle=\color[rgb]{0.133,0.133,0.6},
		commentstyle=\color[rgb]{0.133,0.545,0.133},
		stringstyle=\color[rgb]{0.627,0.126,0.941},
        aboveskip=1.5em,
	}
}




%%%%%%%%%%%%%%%%%%%%%%%%%%%%% Kopf-/Fußzeilenwechsel %%%%%%%%%%%%%%%%%%%%%%%%%%%
\setlength{\headheight}{72pt}

%% Define default layout of all pages
\newcommand{\defaultpagelayout}{
    \fancyhead[L]{
        \ifshowfirmenlogo
            % Firmenlogo und DHBW-Logo --> Firmenlogo links
            \ifshowdhbwlogo
                \iflang{de}{
                    %trim=left bottom right top
                    \includegraphics[height=1.4cm, left, trim=0cm 1.5cm 0cm -1.5cm]{company_de}
                }
                \iflang{en}{
                    \includegraphics[height=1.4cm, left, trim=0cm 1.5cm 0cm -1.5cm]{company_en}
                }
            % Nur Firmenlogo --> \langkopfz links
            \else
                \vspace{0.5cm}\small \langkopfz \vspace{0.08cm}
            \fi
        % Nur DHBW-Logo bzw. kein Logo --> \langkopfz links
        \else
            \vspace{0.5cm}\small \langkopfz \vspace{0.08cm}
        \fi
    }
    \fancyhead[R]{
        \ifshowfirmenlogo
            % Firmenlogo und DHBW-Logo --> DHBW-Logo rechts
            \ifshowdhbwlogo
                \iflang{de}{
                    \includegraphics[height=1.8cm, right, trim=0cm 2cm 0cm -2cm]{dhbw_de.png}
                }
                \iflang{en}{
                    \includegraphics[height=1.8cm, right, trim=0cm 0.5cm 0cm -0.5cm]{dhbw_en.png}
                }
            % Nur Firmenlogo --> Firmenlogo rechts
            \else
                \iflang{de}{
                    \includegraphics[height=1.4cm, right, trim=0cm 1.5cm 0cm -1.5cm]{company_de}
                }
                \iflang{en}{
                    \includegraphics[height=1.4cm, right, trim=0cm 1.5cm 0cm -1.5cm]{company_en}
                }
            \fi
        \else
            % Nur DHBW-Logo --> DHBW-Logo rechts
            \ifshowdhbwlogo
                \iflang{de}{
                    \includegraphics[height=1.8cm, right, trim=0cm 2cm 0cm -2cm]{dhbw_de.png}
                }
                \iflang{en}{
                    \includegraphics[height=1.8cm, right, trim=0cm 0.5cm 0cm -0.5cm]{dhbw_en.png}
                }
            \fi
        \fi
    }

    \renewcommand{\footrulewidth}{0.4pt} % Horizontal line -> default is 0pt
    \fancyfoot[l]{\autor} % \tiny vor \autor einfügen, um Schriftgröße zu verringern
    \fancyfoot[c]{\fusszmitte}
}

%% Only change pagenumbering
\newcommand{\setpagestylehead}{
    \fancypagestyle{plain}{
        \defaultpagelayout{}

        \fancyfoot[r]{\langseite\ \thepage\ \langseitevon\ \pageref*{endOfRomanNumbering}}
    }
    \pagestyle{plain}
    \pagenumbering{roman}
}

\newcommand{\setpagestylecontent}{
    \fancypagestyle{plain}{
        \defaultpagelayout{}

        \fancyfoot[r]{\langseite\ \thepage\ \langseitevon\ \pageref*{endOfArabicNumbering}}    
    }
    \pagestyle{plain}
    \pagenumbering{arabic}
}

\newcommand{\setpagestylefoot}{
    \fancypagestyle{plain}{%
        \defaultpagelayout{}

        \fancyfoot[r]{\langseite\ \thepage\ \langseitevon\ \pageref*{LastPage}}
    }
    \pagestyle{plain}
    \pagenumbering{Alph}
}


%%%%%%%%%%%%%%%%%%%%%%%%%%%%%%%%%%%%%%%%%%%%%%%%%%%%%%%%%%%%%%%%%%%%%%%%%%%%%%%%

% Einstellung der Sprache des Paketes Babel und der Verzeichnisüberschriften

\iflang{de}{
    \usepackage[english, ngerman]{babel}
    \selectlanguage{ngerman}
}
\iflang{en}{
    \usepackage[ngerman, english]{babel}
    \selectlanguage{english}
}

\usepackage[utf8]{inputenc}
\usepackage[T1]{fontenc}
\usepackage{tikz}
%\usepackage{../additionalPackages/tikz-uml} % UML Diagramme
%\usepackage[european]{../additionalPackages/circuitikz}
%%%%%%% Package Includes %%%%%%%

\usepackage[margin=2.5cm,foot=1cm,top=3cm,bottom=3cm]{geometry}	% Seitenränder und Abstände
\usepackage[activate]{microtype} %Zeilenumbruch und mehr
\usepackage[onehalfspacing]{setspace}
\usepackage{makeidx}
\usepackage[autostyle=true,german=quotes]{csquotes}
\usepackage{longtable}
\usepackage{enumitem}	% mehr Optionen bei Aufzählungen
\usepackage{graphicx}
\usepackage{xcolor} 	% für HTML-Notation
\usepackage{float}
\usepackage{array}
\usepackage{calc}		% zum Rechnen (Bildtabelle in Deckblatt)
\usepackage[right]{eurosym}
\usepackage{wrapfig}
\usepackage{pgffor} % für automatische Kapiteldateieinbindung
\usepackage[perpage, hang, multiple, stable]{footmisc} % Fussnoten
\usepackage[absolute]{textpos}
\usepackage[printonlyused]{acronym} % falls gewünscht kann die Option footnote eingefügt werden, dann wird die Erklärung nicht inline sondern in einer Fußnote dargestellt
\usepackage{scrhack} % in Kombination mit listings-Package kommt es zu Warnings, dieses Paket verhindert die Warnings! Ggf. auskommentieren und die Warnings akzeptieren falls Verzeichnisse nicht so dargestellt werden wie gewünscht
\usepackage{listings} % Code-Listings
%\usepackage[numbered, framed]{matlab-prettifier}
\usepackage[framed]{matlab-prettifier} % .sty-Datei muss vorhanden sein! Kann auskommentiert werden, falls keine Matlab-Listings in der Arbeit enthalten sind.
\usepackage{color, colortbl}  %Für Highlighten der Tabellenzeilen
\usepackage{amsmath}% http://ctan.org/pkg/amsmath
\usepackage{amssymb}
\usepackage[export]{adjustbox} % Package, um Bilder in \includegraphics nach links bzw. rechts zu positionieren
\usepackage{pdfpages}         % pdf-Seiten einbinden
\usepackage{wasysym} % für Promille-Zeichen
\usepackage{changepage} % für adjustwidth-Umgebung
\usepackage{makecell} % für Tabellen


% eine Kommentarumgebung "k" (Handhabe mit \begin{k}<Kommentartext>\end{k},
% Kommentare werden rot gedruckt). Wird \% vor excludecomment{k} entfernt,
% werden keine Kommentare mehr gedruckt.
\usepackage{comment}
\specialcomment{k}{\begingroup\color{red}}{\endgroup}
%\excludecomment{k}


%%%%%% Configuration %%%%%

%% Anwenden der Einstellungen

\usepackage{\schriftart}
\ladefarben{}

% Titel, Autor und Datum
\title{\titel}
\author{\autor}
\date{\datum}

%\usepackage[list=true]{subcaption}

% PDF Einstellungen
\usepackage[%
    pdftitle={\titel},
    pdfauthor={\autor},
    pdfsubject={\arbeit},
    pdfcreator={pdflatex, LaTeX with KOMA-Script},
    pdfpagemode=UseOutlines, 		% Beim Oeffnen Inhaltsverzeichnis anzeigen
    pdfdisplaydoctitle=true, 		% Dokumenttitel statt Dateiname anzeigen.
    pdflang={\sprache}, 			% Sprache des Dokuments.
]{hyperref}

% (Farb-)einstellungen für die Links im PDF
\hypersetup{%
    colorlinks=true, 		% Aktivieren von farbigen Links im Dokument
    linkcolor=black, 	    % Farbe festlegen
    citecolor=LinkColor,
    filecolor=LinkColor,
    menucolor=LinkColor,
    urlcolor=LinkColor,
    %linktocpage=true, 		% Nicht der Text sondern die Seitenzahlen in Verzeichnissen klickbar
    linktoc=all,            % Seitenzahlen und Text klickbar
    bookmarksnumbered=true 	% Überschriftsnummerierung im PDF Inhalt anzeigen.
}
% Workaround um Fehler in Hyperref, muss hier stehen bleiben
\usepackage{bookmark} %nur ein latex-Durchlauf für die Aktualisierung von Verzeichnissen nötig

% Schriftart in Captions etwas kleiner
\addtokomafont{caption}{\small}

\usepackage{subfig}

% Literaturverweise (sowohl deutsch als auch englisch)
\iflang{de}{%
    \usepackage[
        backend=biber,          % empfohlen. Falls biber Probleme macht: bibtex
        %bibwarn=true,
        %bibencoding=utf8,	    % wenn .bib in utf8, sonst ascii
        sortlocale=de_DE,
        sorting=none,
        style=\zitierstil,
    ]{biblatex}
}
\iflang{en}{%
    \usepackage[
        backend=biber,		    % empfohlen. Falls biber Probleme macht: bibtex
        %bibwarn=true,
        %bibencoding=utf8,	    % wenn .bib in utf8, sonst ascii
        sortlocale=en_US,
        sorting=none,
        style=\zitierstil,
    ]{biblatex}
}


\ladeliteratur{}

% Glossar
\usepackage[nonumberlist,toc]{glossaries}
\usepackage{blindtext} % Blindtext-Package. Common Usage: \blindtext für einzelnen Abschnitt, \Blindtext für mehrere Abschnitte

%%%%%% Additional settings %%%%%%

% Hurenkinder und Schusterjungen verhindern
% http://projekte.dante.de/DanteFAQ/Silbentrennung
\clubpenalty = 10000 % schließt Schusterjungen aus (Seitenumbruch nach der ersten Zeile eines neuen Absatzes)
\widowpenalty = 10000 % schließt Hurenkinder aus (die letzte Zeile eines Absatzes steht auf einer neuen Seite)
\displaywidowpenalty=10000

\setcounter{biburlnumpenalty}{100}
\setcounter{biburlucpenalty}{100}
\setcounter{biburllcpenalty}{100}

% Bildpfad
\graphicspath{{../content/nonTexFiles/images/}}
%!TEX root = ../../../document/document.tex

\makeatletter

% \saveimage{<ID>}{<imageSetting(e.g. !ht)>}{<graphicSettings(e.g. width=0.7)>}{<imageName>}{<description>}
\newcommand\saveimage[5]{%
    \@namedef{img@#1}{
        \begin{figure}[#2]
            \centering
            \includegraphics[#3]{#4}
            \caption{#5}
            \label{img:#1}
        \end{figure}
    }
}

% \useimage{<ID>}
\newcommand\useimage[1]{%
    \@nameuse{img@#1}
}

\makeatother

%!TEX root = ../../../document/document.tex

% FORMAT: \saveimage{<ID>}{<imageSetting(e.g. !ht)>}{<graphicSettings(e.g. width=0.7)>}{<imageName>}{<description>}


\saveimage{bsp}{h}{width=0.7\textwidth}{dhbw_de.png}{Beispiel: einfügen eines Bildes \cite{2024}} % externe Bilder einbinden

% Einige häufig verwendete Sprachen
\lstloadlanguages{PHP,Python,Java,C,C++,bash}
\listingsettings{}
% Umbennung des Listings
\renewcommand\lstlistingname{\langlistingname}
\renewcommand\lstlistlistingname{\langlistlistingname}
\def\lstlistingautorefname{\langlistingautorefname}
%!TEX root = ../../../document/document.tex

\makeatletter

% \savelisting{<ID>}{<language>}{<caption>}{<fileName>}
\newcommand\savelisting[4]{%
    \@namedef{lst@#1}{
        \lstinputlisting[language={#2}, caption={#3}, label={lst:#1}]{../content/nonTexFiles/code/#4}
    }
}

% \uselisting{<ID>}
\newcommand\uselisting[1]{%
    \@nameuse{lst@#1}
}

\makeatother

%!TEX root = ../../../document/document.tex

% FORMAT: % \savelisting{<ID>}{<language>}{<caption>}{<fileName>}


\savelisting{bsp_code-2}{C++}{Beispiel: indirektes einfügen von Code über externe Datei}{hello-world.cpp} % externe Listings einbinden

% Abstände in Tabellen
\setlength{\tabcolsep}{\spaltenabstand}
\renewcommand{\arraystretch}{\zeilenabstand}

\usepackage{xspace}
\newcommand{\lastcontentpage}{}
\usepackage{amsfonts}

\usetikzlibrary{shapes,arrows,calc}
\usepackage{relsize}

\usepackage{censor}

\usepackage{eso-pic}


%% Paket um Textteile drehen zu können
%\usepackage{rotating}
%% Paket um Seite im Querformat anzuzeigen
%\usepackage{lscape}

\newcommand\Watermark{%
    \put(0,0){%
        \parbox[b][\paperheight]{\paperwidth}{%
            \vfill
            \includepdf[scale=0.8,angle=50,pages={1},pagecommand={}]{../settings/watermark/watermark.pdf}
            \vfill
        }
    }
}

\ifrefWithPages
    %RJG8FE: add a pageref to autoref whenever the referenced page is not the same as the current one
    %        useful for printed documents without clickable hyperlinks
    \AtBeginDocument{\let\oldautoref\autoref}
    \AtBeginDocument{
        \renewcommand{\autoref}[1]{%
            \oldautoref{#1}%
            \ifthenelse{\thepage=\pageref{#1}}% if current page number equals the referenced page number
            {}% then add nothing
            { (S. \pageref{#1})}% else add the text
        }
    }
\fi

\usepackage{amssymb} % Erweiterung der Symbole in Mathematikumgebung

\iflang{de}{\usepackage{icomma}} % Europäiches Komma in Formeln




%selbst hinzugefügt
\usepackage{pdfpages}