%%%%%%%%%%%%%%%%%%%%%%%%%%%%%%%%%%%%%%%%%%%%%%%%%%%%%%%%%%%%%%%%%%%%%%%%%%%%%%%
%                                   Einstellungen
%
% Hier können alle relevanten Einstellungen für diese Arbeit gesetzt werden.
% Dazu gehören Angaben u.a. über den Autor sowie Formatierungen.
%
%
%%%%%%%%%%%%%%%%%%%%%%%%%%%%%%%%%%%%%%%%%%%%%%%%%%%%%%%%%%%%%%%%%%%%%%%%%%%%%%%


%%%%%%%%%%%%%%%%%%%%%%%%%%%%%%%%%%%% Sprache %%%%%%%%%%%%%%%%%%%%%%%%%%%%%%%%%%
%% Aktuell sind Deutsch und Englisch unterstützt.
%% Es werden nicht nur alle vom Dokument erzeugten Texte in
%% der entsprechenden Sprache angezeigt, sondern auch weitere
%% Aspekte angepasst, wie z.B. die Anführungszeichen und
%% Datumsformate.
\setzesprache{de} % de oder en
%%%%%%%%%%%%%%%%%%%%%%%%%%%%%%%%%%%%%%%%%%%%%%%%%%%%%%%%%%%%%%%%%%%%%%%%%%%%%%%


%%%%%%%%%%%%%%%%%%%%%%%%%%%%%%%%%%% Angaben  %%%%%%%%%%%%%%%%%%%%%%%%%%%%%%%%%%
%% Die meisten der folgenden Daten werden auf dem
%% Deckblatt angezeigt, einige auch im weiteren Verlauf
%% des Dokuments.
\setzemartrikelnr{1234567}
\setzekurs{TINF22ITA}
\setzetitel{Thema der Arbeit}
\setzedatumAnfang{01.01.2023}
\setzedatumAbgabe{31.12.2023}
\setzefirma{MAHLE International GmbH}
\setzefirmenort{Stuttgart}
\setzeabgabeort{Stuttgart}
\setzeabschluss{Bachelor of Science}
\setzestudiengang{IT-Automotive}
\setzedhbw{Stuttgart}
\setzehochschule{Duale Hochschule Baden-Württemberg}
\setzebetreuer{B. Sc. Max Mustermann}
\setzegutachter{Prof. Dr. rer. nat. Gustaf Gutachter}
\setzezeitraum{24.04.2023 - 14.08.2023}
\setzearbeit{T3\_1000}
\setzeautor{John Doe}
\setzesemester{4.}
\setzestudienrichtung{Informatik}
\setzejahrgang{2022}
\setzeabteilung{Abteilung}
\setzestandort{Stuttgart}
\setzefusszmitte{\arbeit}

\inhalttrue                 % auskommentieren oder ändern zu \inhaltfalse, falls kein Inhaltsverzeichnis eingefügt werden soll
\unterschriftenblatttrue    % auskommentieren oder ändern zu \unterschriftenblattfalse, falls kein Unterschriftenblatt eingefügt werden soll
\selbsterkltrue             % auskommentieren oder ändern zu \selbsterklfalse, wenn keine Selbstständigkeitserklärung benötigt wird
\sperrvermerktrue           % auskommentieren oder ändern zu \sperrvermerkfalse, wenn kein Sperrvermerk benötigt wird
\abkverztrue                % auskommentieren oder ändern zu \abkverzfalse, wenn kein Abkürzungsverzeichnis benötigt wird
\abbverztrue                % auskommentieren oder ändern zu \abbverzfalse, wenn kein Abbildungsverzeichnis benötigt wird
\tableverztrue              % auskommentieren oder ändern zu \tableverzfalse, wenn kein Tabellenverzeichnis benötigt wird
\listverztrue               % auskommentieren oder ändern zu \listverzfalse, wenn kein Listingsverzeichnis benötigt wird
\formelverztrue             % auskommentieren oder ändern zu \formelverzfalse, wenn kein Formelverzeichnis benötigt wird
\formelgroeverztrue			% auskommentieren oder ändern zu \formelgroeverzfalse, wenn kein Formelgrößenverzeichnis benötigt wird
\abstracttrue               % auskommentieren oder ändern zu \abstractfalse, wenn kein Abstract gewünscht ist
\bothabstractstrue          % auskommentieren oder ändern zu \bothabstractsfalse, wenn nur der Abstract in der Hauptsprache eingefügt werden soll
\appendixtrue               % auskommentieren oder ändern zu \appendixfalse, wenn kein Anhang gewünscht ist
\literaturtrue              % auskommentieren oder ändern zu \literaturfalse, wenn kein Literaturverzeichnis gewünscht ist (\appendixtrue muss gesetzt sein!)
\glossartrue                % auskommentieren oder ändern zu \glossarfalse, wenn kein Glossar gewünscht ist (\appendixtrue muss gesetzt sein!)
\anhangtrue                 % auskommentieren oder ändern zu \anhangfalse, wenn kein Anhang gewünscht ist
%\formblatttrue              % auskommentieren oder ändern zu \formblattfalse, wenn kein Formblatt gewünscht ist

%\refWithPagestrue           % ändern zu \refWithPagestrue, wenn die Seitenzahl bei Verweisen auf Kapitel engefügt werden sollen

\reviewertrue				% auskommentiren oder ändern zu \reviewerfalse wenn kein Gutachter gesetzt werden muss

% Angabe des roten/gelben/grünen Punktes auf dem Titelblatt zur Kennzeichnung der Vertraulichkeitsstufe.
% Mögliche Angaben sind \greendottrue, \yellowdottrue und \reddottrue. Werden mehrere angegeben, wird der Punkt mit der höheren
% Vertraulichkeitsstufe angezeigt.
% Wird keines der Kommandos angegeben, wird kein Punkt gezeichnet.
%\greendottrue
%\yellowdottrue
%\reddottrue

%\watermarktrue              % auskommentieren oder ändern zu \watermarkfalse, wenn kein "Vertraulich"-Wasserzeichen auf den Seiten eingefügt werden soll
%%%%%%%%%%%%%%%%%%%%%%%%%%%%%%%%%%%%%%%%%%%%%%%%%%%%%%%%%%%%%%%%%%%%%%%%%%%%%%%%


%%%%%%%%%%%%%%%%%%%%%%%%%%%% Literaturverzeichnis %%%%%%%%%%%%%%%%%%%%%%%%%%%%%%
%% Bei Fehlern während der Verarbeitung bitte in settings/header/header.tex
%% bei der Einbindung des Pakets biblatex (ungefähr ab Zeile 110,
%% einmal für jede Sprache), biber in bibtex ändern.
\newcommand{\ladeliteratur}{
    \addbibresource{../settings/bibliographie.bib}
}

%% Zitierstil
%% siehe: http://ctan.mirrorcatalogs.com/macros/latex/contrib/biblatex/doc/biblatex.pdf (3.3.1 Citation Styles)
%% mögliche Werte z.B numeric-comp, alphabetic, authoryear
\setzezitierstil{numeric-comp}
%%%%%%%%%%%%%%%%%%%%%%%%%%%%%%%%%%%%%%%%%%%%%%%%%%%%%%%%%%%%%%%%%%%%%%%%%%%%%%%%


%%%%%%%%%%%%%%%%%%%%%%%%%%%%%%%%% Layout %%%%%%%%%%%%%%%%%%%%%%%%%%%%%%%%%%%%%%%
%% Verschiedene Schriftarten
% laut nag Warnung: palatino obsolete, use mathpazo, helvet (option scaled=.95), courier instead
\setzeschriftart{lmodern} % palatino oder goudysans, lmodern, libertine

%% Abstand vor Kapitelüberschriften zum oberen Seitenrand
\setzekapitelabstand{10pt}

%% Spaltenabstand
\setzespaltenabstand{10pt}
%% Zeilenabstand innerhalb einer Tabelle
\setzezeilenabstand{1.5}


%% Flags zum Anzeigen der Bilder

% Logo der Firma im gesamten Dokument anzeigen --> Bild mit Dateinamen
% "company" im images ordner ablegen & Format muss gegebenenfalls in 
% content/additionals/cover-sheet.tex und settings/config/config.tex
% unter "Kopf-/Fußzeilenwechsel" (ca. Z. 200) angepasst werden.
% Ist die Option nicht gesetzt, wird das Logo durch den Text in der Variable
% \setzelangkopfz in settings/lang/de.tex bzw. settings/lang/en.tex ersetzt.
\showfirmenlogotrue                 % auskommentieren oder ändern zu \showfirmenlogofalse, falls kein Logo eingefügt werden soll

% Logo der DHBW im gesamten Dokument anzeigen
\showdhbwlogotrue                   % auskommentieren oder ändern zu \showdhbwlogofalse, falls kein Logo eingefügt werden soll

% Unterschrift als Bild anzeigen --> Bild mit Dateinamen "signature" im 
% images ordner ablegen & Format muss gegebenenfalls in content/additionals/
% declaration.tex und content/additionals/confidentiality-statement.tex
% angepasst werden
\showsignaturetrue                  % auskommentieren oder ändern zu \showsignaturefalse, falls keine Unterschrift eingefügt werden soll
%%%%%%%%%%%%%%%%%%%%%%%%%%%%%%%%%%%%%%%%%%%%%%%%%%%%%%%%%%%%%%%%%%%%%%%%%%%%%%%%